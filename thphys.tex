%---------------------------------------------------
%	PACKAGES AND OTHER DOCUMENT CONFIGURATIONS
%------------------------------------------------------------------------------------


\documentclass[1 pt]{article}
\usepackage{amsmath,amsthm,amssymb}
\usepackage{mathtext}
\usepackage[T1,T2A]{fontenc}
\usepackage[utf8]{inputenc}
\usepackage[english,russian]{babel}
\usepackage{graphicx}
\usepackage{natbib}
\usepackage{pgfplots}
\usepackage[inkscapeformat=png]{svg}
\pgfplotsset{compat=1.9}

\begin {document}

\begin{titlepage}
\newcommand{\HRule}{\rule{\linewidth}{0.3 mm}} % Defines a Hnew command for the horizontal lines, change thickness here

\center % Center everything on the page
 
%----------------------------------------------------------------------------------------
%	HEADING SECTIONS
%----------------------------------------------------------------------------------------

\textsc{\Large Московский физико-технический институт }\\[1.5cm] % Name of your university/college
\textsc{\Large Факультет аэрокосмических технологий}\\[0.5cm] % Major heading such as course name
\textsc{\large Кафедра теоретической физики}\\[0.5cm] % Minor heading such as course title

%----------------------------------------------------------------------------------------
%	TITLE SECTION
%----------------------------------------------------------------------------------------

\HRule \\[0.4cm]
{ \huge \bfseries Конспект теорфиза }\\[0.4cm] % Title of your document
\HRule \\[1.5cm]
 
%----------------------------------------------------------------------------------------
%	AUTHOR SECTION
%----------------------------------------------------------------------------------------

\begin{minipage}{0.4\textwidth}
\begin{flushleft} \large
\emph{Автор:}\\ Артем \textsc{Овчинников} % Your name
\end{flushleft}
\end{minipage}
\begin{minipage}{0.4\textwidth}
\begin{flushright} \large
\emph{Преподаватель:} \\
Эмиль Тофикович \textsc{Ахмедов} % Supervisor's Name
\end{flushright}
\end{minipage}\\[4cm]
%	DATE SECTION
%----------------------------------------------------------------------------------------

{\large \today}\\[2cm] % Date, change the \today to a set date if you want to be precise

%----------------------------------------------------------------------------------------
%	LOGO SECTION
%----------------------------------------------------------------------------------------

 
%----------------------------------------------------------------------------------------

\vfill % Fill the rest of the page with whitespace

\end{titlepage}
\tableofcontents
\newpage
\section{23.04.24: Дифференцирование тензоров}
\begin{equation}
    rot \frac{\overline{x}}{(\overline{k}, \overline{x})} = ?, \; \overline{k} = \overline{const}
\end{equation}
\begin{equation*}
    rot \frac{\overline{x}}{(\overline{k}, \overline{x})} = e_{ijk} d_{j} \frac{x_k}{k_l x_l} = e_{ijk} \frac{k_n x_n \cdot d_j x_k - x_k k_t d_j x_t }{(k_l x_l)^2} = e_{ijk} \frac{k_n x_n \delta_{jk} - x_k k_t \delta_{jt}}{(\overline{k}, \overline{x})^2}
\end{equation*}
\begin{equation*}
    e_{ijk} \delta_{jk} = 0 \; (j=k, e_{ijj}=0)
\end{equation*}
\begin{equation*}
    e_{ijk} \delta_{jt} = e_{itk} \; (j=t)
\end{equation*}
\begin{equation*}
    rot \frac{\overline{x}}{(\overline{k}, \overline{x})} = e_{itk} \frac{ - x_k k_t}{(\overline{k}, \overline{x})^2} = e_{ikt} \frac{ x_k k_t}{(\overline{k}, \overline{x})^2} = \frac{[x, k]}{(\overline{k}, \overline{x})^2}
\end{equation*}
\begin{equation}
    rot \frac{\overline{x}}{(\overline{k}, \overline{x})} = \frac{[x, k]}{(\overline{k}, \overline{x})^2}
\end{equation}
\begin{equation}
    rot rot \frac{\overline{k}}{|\overline{r}|} = ?, \; \overline{k} = \overline{const}
\end{equation}

\section{07.05.24: Квантовый осциллятор}
\begin{equation}
    \frac{1}{2}(x^2-\frac{d^2}{dx^2}) \Psi(x) = E \Psi(x), \; \int_{- \infty}^{+ \infty} dx |\Psi(x)|^2 = 1, \; \Psi(x) = ?, \; E = ?
\end{equation}
\begin{equation*}
    a = \frac{1}{\sqrt{2}}(x+\frac{d}{dx})
\end{equation*}
\begin{equation*}
    a^+ = \frac{1}{\sqrt{2}}(x-\frac{d}{dx})
\end{equation*}
\begin{equation*}
    [a^+, a] = a^+ a - aa^+ = \frac{1}{2}((x-\frac{d}{dx})(x+\frac{d}{dx})-(x+\frac{d}{dx})(x-\frac{d}{dx}))
\end{equation*}
\begin{equation*}
    [a^+, a] = \frac{1}{2}(xx+x\frac{d}{dx}-x\frac{d}{dx}-1-\frac{d^2}{dx^2}-xx+x\frac{d}{dx}-x\frac{d}{dx}-1+\frac{d^2}{dx^2})
\end{equation*}
\begin{equation*}
    [a^+, a] = -1
\end{equation*}
\begin{equation*}
    a^+a+aa^+ = \frac{1}{2}(xx+x\frac{d}{dx}-x\frac{d}{dx}-1-\frac{d^2}{dx^2}+xx-x\frac{d}{dx}+x\frac{d}{dx}+1-\frac{d^2}{dx^2})
\end{equation*}
\begin{equation*}
    a^+a+aa^+ = x^2 - \frac{d^2}{dx^2}
\end{equation*}
\begin{equation*}
    H = \frac{1}{2}(a^+a+aa^+) = \frac{1}{2}(2a^+a+1) = a^+a+\frac{1}{2}
\end{equation*}
\begin{equation*}
    (a^+a+\frac{1}{2}) \Psi(x) = E \Psi(x)
\end{equation*}
\begin{equation*}
    a \Psi_0(x) = 0
\end{equation*}
\begin{equation}
    \frac{1}{2} \Psi(x) = E \Psi(x), \; E_0 = \frac{1}{2}
\end{equation}
\begin{equation*}
    \frac{1}{\sqrt{2}}(x+\frac{d}{dx}) \Psi_0(x) = 0
\end{equation*}
\begin{equation}
    \Psi_0(x) = A \cdot exp(-\frac{x^2}{2})
\end{equation}
\begin{equation*}
    x \cdot exp(-\frac{x^2}{2})+\frac{d}{dx} exp(-\frac{x^2}{2}) = 0
\end{equation*}
\begin{equation*}
    x \cdot exp(-\frac{x^2}{2})-exp(-\frac{x^2}{2})x = 0
\end{equation*}
\begin{equation*}
    \int_{- \infty}^{+ \infty} dx |A \cdot exp(-\frac{x^2}{2})|^2 = 1
\end{equation*}
\begin{equation}
    \int_{- \infty}^{+ \infty} dx |exp(-x^2)| = \frac{1}{A^2} = \sqrt{\pi}, \; A_0=\frac{1}{\pi^{1/4}}
\end{equation}
\begin{equation*}
    \Psi_1(x) = a^+ \Psi_0(x) = \frac{1}{\sqrt{2}}(x-\frac{d}{dx}) \frac{1}{\pi^{1/4}} \cdot exp(-\frac{x^2}{2})
\end{equation*}
\begin{equation}
    \Psi_1(x) = A_1 a^+ \Psi_0(x) = A_1 \frac{\sqrt{2}}{\pi^{1/4}} \cdot exp(-\frac{x^2}{2}) x, \; A_1 = ?
\end{equation}
\begin{equation*}
    (a^+a+\frac{1}{2}) \Psi_1(x) = E_1 \Psi_1 (x)
\end{equation*}
\begin{equation*}
    (a^+a+\frac{1}{2}) a^+ \Psi_0(x) = E_1 a^+ \Psi_0(x)
\end{equation*}
\begin{equation*}
    a^+a a^+ = a^+ (a^+ a + 1)
\end{equation*}
\begin{equation*}
    a^+ (a^+ a + 1) \Psi_0(x)+\frac{1}{2} a^+ \Psi_0(x) = E_1 a^+ \Psi_0(x)
\end{equation*}
\begin{equation}
    a \Psi_0(x) = 0 \Rightarrow E_1 = 1+ \frac{1}{2} = \frac{3}{2}
\end{equation}
\begin{equation}
    \Psi_n(x) = A_n (a^+)^n \Psi_0(x), \; E_n = ?
\end{equation}
\begin{equation*}
    (a^+a+\frac{1}{2}) \Psi_{n+1}(x) = E_{n+1} \Psi_{n+1} (x), \; \Psi_{n+1}(x) = A_{n+1} a^+ \Psi_n (x)
\end{equation*}
\begin{equation*}
    (a^+a+\frac{1}{2}) a^+ \Psi_n (x) = E_{n+1} a^+ \Psi_n (x)
\end{equation*}
\begin{equation*}
    (a^+a+\frac{1}{2}) \Psi_{n}(x) = E_{n} \Psi_{n} (x)
\end{equation*}
\begin{equation*}
    a^+ (a^+ a + 1) \Psi_n (x)+\frac{1}{2} a^+ \Psi_n (x) = E_{n+1} a^+ \Psi_n (x)
\end{equation*}
\begin{equation*}
    a^+ (E_n + \frac{1}{2}) \Psi_n (x)+\frac{1}{2} a^+ \Psi_n (x) = E_{n+1} a^+ \Psi_n (x)
\end{equation*}
\begin{equation*}
    E_n + 1 = E_{n+1}
\end{equation*}
\begin{equation*}
    E_n = E_0 + n
\end{equation*}
\begin{equation}
    E_n = \frac{1}{2} + n
\end{equation}

\end{document}
